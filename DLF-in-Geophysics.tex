\documentclass[a4paper, twoside, parskip, 10pt]{scrartcl}

% Packages
\usepackage[USenglish]{babel}
\usepackage[utf8]{inputenc}
\usepackage{lmodern}
\usepackage[T1]{fontenc}

\usepackage[hmargin=1.5cm, top=1.5cm, bottom=0cm, includeheadfoot]{geometry}
\usepackage{multicol}
\usepackage{fancyhdr}
\usepackage{xspace}
\usepackage{amssymb, amsmath, amsfonts}
\usepackage{natbib}

\usepackage[pdftex, final]{hyperref}

% Settings
\hypersetup{allcolors=blue, allbordercolors={0 0 .5}, colorlinks=true}

\frenchspacing

\setkomafont{subsection}{\normalfont\itshape}

\pagestyle{fancy}
\fancyhf{}
\renewcommand{\headrulewidth}{0.0pt}
\renewcommand{\footrulewidth}{0.0pt}
\addtolength{\headheight}{0.0pt}
\fancyhead[LO, RE]{Digital Linear Filters in Geophysics}
\fancyhead[RO, LE]{\thepage}
\fancypagestyle{empty}{
  \fancyhead[LE, RE, LO]{}
  \fancyhead[RO]{\today}}

\setcounter{secnumdepth}{-1}

\setlength\columnsep{15pt}
% \columnseprule1pt

% Commands
\newcommand{\mr}[1]{\mathrm{#1}}

\begin{document}
\thispagestyle{empty}

{\huge \bf \sffamily Digital Linear Filters in Geophysics\\[.5cm]}
{\small\emph{Excerpt from \cite{GEO.18.Werthmuller}; for the full article see
\href{https://github.com/empymod/article-fdesign}{github.com/empymod/article-fdesign}.}}

\vspace{.5cm}

\begin{multicols}{2}

\section{Review}

In his Ph.D. thesis, \cite{PhD.70.Ghosh} proposed a linear filter method for
the numerical evaluation of Hankel transforms that subsequently revolutionized
the computation of electromagnetic (EM) responses in the field of geophysical
exploration. If you use a code that calculates EM responses in the
wavenumber-frequency domain and transforms them to the space-frequency domain,
chances are high that it uses the \emph{digital linear filter} (DLF) method.
Most practical 1D EM modeling codes rely on the DLF method for rapid
computations; these codes are not only used for standalone simulations of EM
fields in layered 1D media, but they are also commonly embedded within 3D
modeling codes for computing the primary fields in scattered-field formulations
or for the background fields required by integral equation methods. Thus, the
DLF method is an important component of many commonly used modeling codes for
EM geophysical data.

\cite{GP.71.Ghosh} states that the DLF idea is based on suggestions made four
decades earlier by \cite{PHY.33.Slichter} and \cite{GEO.40.Pekeris}, in that
``\emph{the kernel function is dependent only on the layer parameters, and an
expression relating it to the field measurements can be obtained by
mathematical processes.}'' However, until the introduction of DLF, these
suggestions found no widespread use, likely because of the missing computer
power to calculate the filter coefficients. He further states that credit goes
to \cite{BK.68.Koefoed, GP.70.Koefoed}, who retook the task of direct
interpretational methods with the introduction of raised kernel functions. DLF
is, as such, an improvement of that approach, providing a faster and simpler
method.

The DLF technique is sometimes referred to as the \emph{fast Hankel transform}
(FHT), popular because of the similarity of the name to the well known
\emph{fast Fourier transform} (FFT), although the algorithms for these
techniques are completely different. The FHT name was likely inspired by the
title of a paper by \cite{GP.79.Johansen}. However, the name FHT can be
misleading as it has \emph{Hankel} in the name, whereas the DLF approach can
more generally be applied to other linear transforms, for example Fourier sine
and cosine transforms.

The introduction of linear filters to EM geophysics, in parallel with rising
computational power, initiated a wealth of investigations, leading to the
development of computer programs that extended and improved the DLF method, and
to numerous publications. These publications fall broadly into one or several
of three categories: (1) new applications that extend the usage of DLF to other
EM measurement techniques; (2) filter improvements that provide either new
filters or improved methods for the determination of filter coefficients; and
(3) computational tools that compute EM responses using DLF techniques. Here,
we briefly review the most relevant publications, without claiming
completeness.

\subsection{(1) New applications}

Ghosh used the method originally for the computation of type curves for
Schlumberger and Wenner resistivity soundings: \cite{GP.71.Ghosh} derives a
resistivity model from given Schlumberger or Wenner sounding curves; and
\citet{GP.71.Ghoshb} provides filters for the inverse operation, deriving
resistivity sounding curves from a given resistivity model. The method was next
applied to electromagnetic soundings with horizontal and perpendicular coils
\citep{GP.72.Koefoed}, to vertical coplanar coil systems \citep{GP.73.Verma},
to dipoles and other two electrode systems \citep{GP.74.Das, GP.74.Dasb,
GP.80.Das, GEO.94.Sorensen}, and to vertical dikes, hence vertical instead of
horizontal layers \citep{GEO.75.Niwas}. The first filters were specific to a
particular resistivity sounding type and its transform; later publications used
the method to get one type curve from another type curve \citep{GP.77.Kumar,
GP.78.Kumar}, or generalized the method to be applicable to a wider set of
problems \citep{EXG.80.Davis, GXP.81.Das, GEO.84.Das, GP.84.ONeill}.
Eventually, it passed from pure layered modeling to primary-secondary field
formulations for 3D problems, where DLF is used to compute the spatial
Fourier-Hankel transforms in a horizontally layered background medium and to
compute transient responses from frequency domain computations
\citep{GJI.81.Das, GJI.82.Das, GEO.84.Anderson, GEO.86.Newman,
MGS.17.Kruglyakov}. Other publications delved into the theory of the method,
analyzing the oscillating behaviour of the filters and trying to estimate the
error of DLF \citep{GP.72.Koefoedb, GP.76.Koefoedb, GP.79.Johansen,
GP.90.Christensen}.

\subsection{(2) Filter improvements}

\cite{PhD.70.Ghosh} derived the filter coefficients in the spectral domain by
dividing the output spectrum by the input spectrum followed by an inverse
Fourier transform. Improvements to the determination of filter coefficients
were provided by \cite{EXG.75.ONeill, GEO.77.Nyman, GEO.82.Das}, or
specifically for the Fourier transform by \cite{GP.86.Nissen}. A direct
integration method was used by \cite{GP.76.Bichara} and \cite{GP.78.Bernabini}.
\cite{GP.79.Koefoed} proposed a Wiener-Hopf least-squares method, which was
further improved by many authors \citep{GP.82.Guptasarma, GEO.1982.Murakami,
GP.97.Guptasarma}. \cite{GP.07.Kong} proposed a direct matrix inversion method
to solve the convolution equation, which requires only the input and output
sample values. To evaluate the filter's effectiveness, he defines a good filter
as one that recovers small or weak diffusive EM fields. This method was also
used by \cite{GEO.09.Key, GEO.12.Key} to create a suite of filters accurate for
marine EM data. Most works have published filters for the Hankel transform with
$J_0$ and $J_1$ Bessel functions (or $J_{-1/2}$, $J_{1/2}$ if applied to the
Fourier sine/cosine transform), as all higher Bessel functions can be
rewritten, via recurrence relations, using only these two. \cite{GP.94.Mohsen}
is one of the rare cases which provides $J_2$ filter weights.

\subsection{(3) Computational tools}

The most well-known codes are likely Anderson's freely available ones.
\cite{USGS.73.Anderson} extends the method to transient responses, applying DLF
not only to the Hankel transform, but also to the Fourier transform. A
transient signal can therefore be obtained be applying twice a digital filter
to the wavenumber-frequency domain calculation. \cite{USGS.75.Anderson,
GEO.79.Anderson} presents improved filters for both Fourier and Hankel
transforms, introducing measures to significantly speed-up the calculation,
such as the lagged convolution or using the same abscissae for $J_0$ and $J_1$.
\cite{TMS.82.Anderson} and the included 801\,pt filter became the de facto
industry standard, to which subsequent filters were compared.
\cite{GEO.89.Anderson} presents a hybrid solution that uses either the DLF or a
much slower but highly accurate adaptive-quadrature approach, which can also be
used to measure the accuracy of the DLF. \cite{GEO.12.Key} presented codes for
comparing the DLF approach with a speed optimized quadrature method called
quadrature-with-extrapolation (QWE); despite its optimizations, QWE is still
not as fast as DLF, but it remains valuable as an independent tool for testing
the accuracy of particular filters for the DLF method. Other examples include
the codes by \cite{GP.75.Johansen}, an interactive system for interpretation of
resistivity soundings, and a tool to calculate filter coefficients by
\cite{GP.90.Christensen}. The latter is available upon request and was used,
for instance, in all the open-source modeling and inversion routines of CSIRO
in the Amira Project 223 \citep{ASEG.07.Raiche}.

All mentioned publications have in common that they were derived for direct
current methods (DC) or low frequency methods, such as time-domain shallow EM
methods (TEM), or controlled-source electromagnetics (CSEM), but not for high
frequency methods such as ground-pe\-ne\-tra\-ting radar (GPR). Generally it
was even thought that the filter method works only in the diffusive limit where
the quasistatic approximation is valid \citep[e.g.,][]{GEO.15.Hunziker}.
Nevertheless, \cite{GEO.17.Werthmuller} tested DLF for modeling 250 MHz
center-frequency GPR data using a 401 pt filter that was designed for diffusive
EM, and found that it was orders of magnitude faster than quadrature
approaches. However, good quality results were only obtained for the first
arrivals at short time-offsets and the later arrivals had poor quality. Yet,
this promising initial test suggests it might be possible to specifically
design a special filter for GPR frequencies where the EM fields propagate as
waves rather than by diffusion.

\section{Theory}
Most of the articles mentioned in the review have detailed derivations of the
digital filter method. In this article we focus on the algorithm, and summarize
the theory only very briefly by following \cite{GEO.12.Key}. In
electromagnetics we often have to evaluate integrals of the form
%
\begin{equation}
  F(r) = \int^\infty_0 f(l)K(l r)\mr{d}l \ ,
  \label{eq:HankelInt}
\end{equation}
%
where $l$ and $r$ denote input and output evaluation values, respectively, and
$K$ is the kernel function. In the specific case of the Hankel transform $l$
corresponds to wavenumber, $r$ to offset, and $K$ to Bessel functions; in the
case of the Fourier transform $l$ corresponds to frequency, $r$ to time, and
$K$ to sine or cosine functions. In both cases it is an infinite integral which
numerical integration is very time-consuming because of the slow decay of the
kernel function and its oscillatory behaviour.

By substituting $r = e^x$ and $l = e^{-y}$ we get
%
\begin{equation}
  e^x F(e^x) = \int^\infty_{-\infty} f(e^{-y})K(e^{x-y})e^{x-y}\mr{d}y\ .
  \label{eq:filtint}
\end{equation}
%
This can be re-written as a convolution integral and be approximated for an
$N$-point filter by
%
\begin{equation}
  F(r) \approx \sum^N_{n=1} \frac{f(b_n/r) h_n}{r}\ ,
  \label{eq:filtapprox}
\end{equation}
%
where $h$ is the digital linear filter, and the logarithmically spaced filter
abscissae is a function of the spacing $\Delta$ and the shift $\delta$,
%
\begin{equation}
  b_n = \exp\left\{\Delta(-\lfloor{(N+1)/2\rfloor}+n) + \delta\right\} \ .
  \label{eq:base}
\end{equation}
%
From equation~\ref{eq:filtapprox} it can be seen that the filter method
requires $N$ evaluations at each $r$. For example, to calculate the frequency
domain result for 100 offsets with a 201\,pt filter requires 20,100 evaluations
in the wavenumber domain. This is why the DLF often uses interpolation to
minimize the required evaluations, either for $F(r)$ in what is referred to as
\emph{lagged convolution DLF}, or for $f(l)$, which we here call \emph{splined
DLF}.

In the direct matrix inversion method for solving for the digital filter
coefficients, equation~\ref{eq:filtapprox} is cast as a linear system,
where $\Delta$ and $\delta$ are preassigned scalar values and
$r$ is a range of $M$ preassigned values $r_m$. The filter base coefficients
$b_n$ are computed using equation~\ref{eq:base} and an array of values for $l$
are computed using $l_{mn} = b_n/r_m$. The linear system's left hand side (LHS)
matrix $\bf{A}$ has dimensions $M \times N$ with coefficients
$A_{mn} = f(l_{mn})/r_m$ and the right hand side (RHS)
vector $\bf{v}$ has elements $v_m = F(r_m)$. The
resulting linear system
%
\begin{equation}
  \bf{A}\bf{h} = \bf{v}
  \label{eq:,linearSystem}
\end{equation}
%
can be solved by direct matrix inversion, or any other matrix inversion
routine, to obtain the vector of filter coefficients $\bf{h}$. For a given
filter length $N$, there are many subjective choices that go into the practical
implementation of this method, including the choice of the transform pairs
$F(r)$ and $f(l)$, as well as the value for $M$ and the particular range and
spacing of the values $r_m$. Once values are chosen for these variables, an
optimal filter can be found by a grid search over $\Delta$ and $\delta$ for
values that produce a high quality filter. The choice of metric for what
constitutes a high quality filter is also subjective and is further discussed
below.


\begin{thebibliography}{}
\itemsep0pt

\bibitem[Anderson, 1973]{USGS.73.Anderson}
Anderson, W.~L.,  1973, Fortran {IV} programs for the determination of the
  transient tangential electric field and vertical magnetic field about a
  vertical magnetic dipole for an m-layer stratified earth by numerical
  integration and digital linear filtering: USGS, {\bf PB221240}.
\newblock
  (\href{https://ntrl.ntis.gov/NTRL/dashboard/searchResults/titleDetail/PB221240.xhtml}{https://ntrl.ntis.gov\-/NTRL\-/dashboard\-/searchResults\-/titleDetail\-/PB221240.xhtml}).

\bibitem[Anderson, 1975]{USGS.75.Anderson}
--------, 1975, Improved digital filters for evaluating {F}ourier and {H}ankel
  transform integrals: USGS, {\bf PB242800}.
\newblock
  (\href{https://pubs.er.usgs.gov/publication/70045426}{https://pubs.er.usgs.gov/publication/70045426}).

\bibitem[Anderson, 1979]{GEO.79.Anderson}
--------, 1979, Numerical integration of related {H}ankel transforms of orders
  0 and 1 by adaptive digital filtering: Geophysics, {\bf 44}, 1287--1305.
\newblock (\href{https://doi.org/10.1190/1.1441007}{doi: 10.1190/1.1441007}).

\bibitem[Anderson, 1982]{TMS.82.Anderson}
--------, 1982, Fast {H}ankel transforms using related and lagged convolutions:
  ACM Trans. Math. Softw., {\bf 8}, 344--368.
\newblock (\href{https://doi.org/10.1145/356012.356014}{doi:
  10.1145/356012.356014}).

\bibitem[Anderson, 1984]{GEO.84.Anderson}
--------, 1984, {On: “Numerical integration of related Hankel transforms by
  quadrature and continued fraction expansion” by \cite{GEO.83.Chave}}:
  Geophysics, {\bf 49}, 1811--1812.
\newblock (\href{https://doi.org/10.1190/1.1441595}{doi: 10.1190/1.1441595}).

\bibitem[Anderson, 1989]{GEO.89.Anderson}
--------, 1989, A hybrid fast {H}ankel transform algorithm for electromagnetic
  modeling: Geophysics, {\bf 54}, 263--266.
\newblock (\href{https://doi.org/10.1190/1.1442650}{doi: 10.1190/1.1442650}).

\bibitem[Bernabini and Cardarelli, 1978]{GP.78.Bernabini}
Bernabini, M., and E. Cardarelli,  1978, The use of filtered {B}essel functions
  in direct interpretation of geoelectrical soundings: Geophysical Prospecting,
  {\bf 26}, 841--852.
\newblock (\href{https://doi.org/10.1111/j.1365-2478.1978.tb01636.x}{doi:
  10.1111/j.1365-2478.1978.tb01636.x}).

\bibitem[Bichara and Lakshmanan, 1976]{GP.76.Bichara}
Bichara, M., and J. Lakshmanan,  1976, Fast automatic processing of resistivity
  soundings Geophysical Prospecting, {\bf 24}, 354--370.
\newblock (\href{https://doi.org/10.1111/j.1365-2478.1976.tb00932.x}{doi:
  10.1111/j.1365-2478.1976.tb00932.x}).

\bibitem[Chave, 1983]{GEO.83.Chave}
Chave, A.~D.,  1983, Numerical integration of related {H}ankel transforms by
  quadrature and continued fraction expansion: Geophysics, {\bf 48},
  1671--1686.
\newblock (\href{https://doi.org/10.1190/1.1441448}{doi: 10.1190/1.1441448}).

\bibitem[Christensen, 1990]{GP.90.Christensen}
Christensen, N.~B.,  1990, Optimized fast {H}ankel transform filters:
  Geophysical Prospecting, {\bf 38}, 545--568.
\newblock (\href{https://doi.org/10.1111/j.1365-2478.1990.tb01861.x}{doi:
  10.1111/j.1365-2478.1990.tb01861.x}).

\bibitem[Das, 1982]{GEO.82.Das}
Das, U.~C.,  1982, Designing digital linear filters for computing resistivity
  and electromagnetic sounding curves: Geophysics, {\bf 47}, 1456--1459.
\newblock (\href{https://doi.org/10.1190/1.1441295}{doi: 10.1190/1.1441295}).

\bibitem[Das, 1984]{GEO.84.Das}
--------, 1984, A single digital linear filter for computations in electrical
  methods--a unifying approach: Geophysics, {\bf 49}, 1115--1118.
\newblock (\href{https://doi.org/10.1190/1.1441726}{doi: 10.1190/1.1441726}).

\bibitem[Das and Ghosh, 1974]{GP.74.Das}
Das, U.~C., and D.~P. Ghosh,  1974, The determination of filter coefficients
  for the computation of standard curves for dipole resistivity sounding over
  layered earth by linear digital filtering: Geophysical Prospecting, {\bf 22},
  765--780.
\newblock (\href{https://doi.org/10.1111/j.1365-2478.1974.tb00117.x}{doi:
  10.1111/j.1365-2478.1974.tb00117.x}).

\bibitem[Das et~al., 1974]{GP.74.Dasb}
Das, U.~C., D.~P. Ghosh, and D.~T. Biewinga,  1974, Transformation of dipole
  resistivity sounding measurements over layered earth by linear digital
  filtering: Geophysical Prospecting, {\bf 22}, 476--489.
\newblock (\href{https://doi.org/10.1111/j.1365-2478.1974.tb00100.x}{doi:
  10.1111/j.1365-2478.1974.tb00100.x}).

\bibitem[Das and Verma, 1980]{GP.80.Das}
Das, U.~C., and S.~K. Verma,  1980, Digital linear filter for computing type
  curves for the two-electrode system of resistivity sounding: Geophysical
  Prospecting, {\bf 28}, 610--619.
\newblock (\href{https://doi.org/10.1111/j.1365-2478.1980.tb01246.x}{doi:
  10.1111/j.1365-2478.1980.tb01246.x}).

\bibitem[Das and Verma, 1981a]{GJI.81.Das}
--------, 1981a, Numerical considerations on computing the {EM} response of
  three-dimensional inhomogeneities in a layered earth: Geophysical Journal
  International, {\bf 66}, 733--740.
\newblock (\href{https://doi.org/10.1111/j.1365-246X.1981.tb04897.x}{doi:
  10.1111/j.1365-246X.1981.tb04897.x}).

\bibitem[Das and Verma, 1981b]{GXP.81.Das}
--------, 1981b, The versatility of digital linear filters used in computing
resistivity and EM sounding curves: Geoexploration, {\bf 18}, 297--310.
\newblock (\href{https://doi.org/10.1016/0016-7142(81)90059-4}{doi:
10.1016/0016-7142(81)90059-4}).

\bibitem[Das and Verma, 1982]{GJI.82.Das}
--------, 1982, Electromagnetic response of an arbitrarily shaped
  three-dimensional conductor in a layered earth — numerical results:
  Geophysical Journal International, {\bf 69}, 55--66.
\newblock (\href{https://doi.org/10.1111/j.1365-246X.1982.tb04935.x}{doi:
  10.1111/j.1365-246X.1982.tb04935.x}).

\bibitem[Davis et~al., 1980]{EXG.80.Davis}
Davis, P.~A., S.~A. Greenhalgh, and N.~P. Merrick,  1980, Resistivity sounding
  computations with any array using a single digital filter: Exploration
  Geophysics, {\bf 11}, 54--62.
\newblock (\href{https://doi.org/10.1071/EG980054}{doi: 10.1071/EG980054}).

\bibitem[Ghosh, 1970]{PhD.70.Ghosh}
Ghosh, D.~P.,  1970, The application of linear filter theory to the direct
  interpretation of geoelectrical resistivity measurements: {Ph.D. Thesis}, {TU
  Delft}.
\newblock
  (\href{http://resolver.tudelft.nl/uuid:88a568bb-ebee-4d7b-92df-6639b42da2b2}{uuid:
  88a568bb-ebee-4d7b-92df-6639b42da2b2}).

\bibitem[Ghosh, 1971a]{GP.71.Ghosh}
--------, 1971a, The application of linear filter theory to the direct
  interpretation of geoelectrical resistivity sounding measurements:
  Geophysical Prospecting, {\bf 19}, 192--217.
\newblock (\href{https://doi.org/10.1111/j.1365-2478.1971.tb00593.x}{doi:
  10.1111/j.1365-2478.1971.tb00593.x}).

\bibitem[Ghosh, 1971b]{GP.71.Ghoshb}
--------, 1971b, Inverse filter coefficients for the computation of apparent
  resistivity standard curves for a horizontally stratified earth: Geophysical
  Prospecting, {\bf 19}, 769--775.
\newblock (\href{https://doi.org/10.1111/j.1365-2478.1971.tb00915.x}{doi:
  10.1111/j.1365-2478.1971.tb00915.x}).

\bibitem[Guptasarma, 1982]{GP.82.Guptasarma}
Guptasarma, D.,  1982, Optimization of short digital linear filters for
  increased accuracy: Geophysical Prospecting, {\bf 30}, 501--514.
\newblock (\href{https://doi.org/10.1111/j.1365-2478.1982.tb01320.x}{doi:
  10.1111/j.1365-2478.1982.tb01320.x}).

\bibitem[Guptasarma and Singh, 1997]{GP.97.Guptasarma}
Guptasarma, D., and B. Singh,  1997, New digital linear filters for {H}ankel
  {J}0 and {J}1 transforms: Geophysical Prospecting, {\bf 45}, 745--762.
\newblock (\href{https://doi.org/10.1046/j.1365-2478.1997.500292.x}{doi:
  10.1046/10.1046/j.1365-2478.1997.500292.x}).

\bibitem[Hunziker et~al., 2015]{GEO.15.Hunziker}
Hunziker, J., J. Thorbecke, and E. Slob,  2015, The electromagnetic response in
  a layered vertical transverse isotropic medium: {A} new look at an old
  problem: Geophysics, {\bf 80}, no. 1, F1--F18.
\newblock (\href{https://doi.org/10.1190/geo2013-0411.1}{doi:
  10.1190/geo2013-0411.1}).

\bibitem[Johansen, 1975]{GP.75.Johansen}
Johansen, H.~K.,  1975, An interactive computer/graphic-display-terminal system
  for interpretation of resistivity soundings: Geophysical Prospecting, {\bf
  23}, 449--458.
\newblock (\href{https://doi.org/10.1111/j.1365-2478.1975.tb01541.x}{doi:
  10.1111/j.1365-2478.1975.tb01541.x}).

\bibitem[Johansen and Sørensen, 1979]{GP.79.Johansen}
Johansen, H.~K., and K. Sørensen,  1979, Fast {H}ankel transforms: Geophysical
  Prospecting, {\bf 27}, 876--901.
\newblock (\href{https://doi.org/10.1111/j.1365-2478.1979.tb01005.x}{doi:
  10.1111/j.1365-2478.1979.tb01005.x}).

\bibitem[Key, 2009]{GEO.09.Key}
Key, K.,  2009, {1D} inversion of multicomponent, multifrequency marine {CSEM}
  data: {M}ethodology and synthetic studies for resolving thin resistive
  layers: Geophysics, {\bf 74}, no. 2, F9--F20.
\newblock (\href{https://doi.org/10.1190/1.3058434}{doi: 10.1190/1.3058434}).

\bibitem[Key, 2012]{GEO.12.Key}
--------, 2012, Is the fast {H}ankel transform faster than quadrature?:
  Geophysics, {\bf 77}, no. 3, F21--F30.
\newblock (\href{https://doi.org/10.1190/GEO2011-0237.1}{doi:
  10.1190/GEO2011-0237.1}).

\bibitem[Koefoed, 1968]{BK.68.Koefoed}
Koefoed, O.,  1968, The application of the kernel function in interpreting
  geoelectrical resistivity measurements: Gebrüder Borntraeger.
\newblock (\href{https://isbnsearch.org/isbn/9783443130022}{{I}SBN:
  9783443130022}).

\bibitem[Koefoed, 1970]{GP.70.Koefoed}
--------, 1970, A fast method for determining the layer distribution from the
  raised kernel function in geoelectrical sounding: Geophysical Prospecting,
  {\bf 18}, 564--570.
\newblock (\href{https://doi.org/10.1111/j.1365-2478.1970.tb02129.x}{doi:
  10.1111/j.1365-2478.1970.tb02129.x}).

\bibitem[Koefoed, 1972]{GP.72.Koefoedb}
--------, 1972, A note on the linear filter method of interpreting resistivity
  sounding data: Geophysical Prospecting, {\bf 20}, 403--405.
\newblock (\href{https://doi.org/10.1111/j.1365-2478.1972.tb00643.x}{doi:
  10.1111/j.1365-2478.1972.tb00643.x}).

\bibitem[Koefoed, 1976]{GP.76.Koefoedb}
--------, 1976, Error propagation and uncertainty in the interpretation of
  resistivity sounding data: Geophysical Prospecting, {\bf 24}, 31--48.
\newblock (\href{https://doi.org/10.1111/j.1365-2478.1976.tb00383.x}{doi:
  10.1111/j.1365-2478.1976.tb00383.x}).

\bibitem[Koefoed and Dirks, 1979]{GP.79.Koefoed}
Koefoed, O., and F.~J.~H. Dirks,  1979, Determination of resistivity sounding
  filters by the {W}iener-{H}opf least-squares method: Geophysical Prospecting,
  {\bf 27}, 245--250.
\newblock (\href{https://doi.org/10.1111/j.1365-2478.1979.tb00968.x}{doi:
  10.1111/j.1365-2478.1979.tb00968.x}).

\bibitem[Koefoed et~al., 1972]{GP.72.Koefoed}
Koefoed, O., D.~P. Ghosh, and G.~J. Polman,  1972, Computation of type curves
  for electromagnetic depth sounding with a horizontal transmitting coil by
  means of a digital linear filter: Geophysical Prospecting, {\bf 20},
  406--420.
\newblock (\href{https://doi.org/10.1111/j.1365-2478.1972.tb00644.x}{doi:
  10.1111/j.1365-2478.1972.tb00644.x}).

\bibitem[Kong, 2007]{GP.07.Kong}
Kong, F.~N.,  2007, Hankel transform filters for dipole antenna radiation in a
  conductive medium: Geophysical Prospecting, {\bf 55}, 83--89.
\newblock (\href{https://doi.org/10.1111/j.1365-2478.2006.00585.x}{doi:
  10.1111/j.1365-2478.2006.00585.x}).

\bibitem[Kruglyakov and Bloshanskaya, 2017]{MGS.17.Kruglyakov}
Kruglyakov, M., and L. Bloshanskaya,  2017, High-performance parallel solver
  for integral equations of electromagnetics based on {G}alerkin method:
  Mathematical Geosciences, {\bf 49}, 751--776.
\newblock (\href{https://doi.org/10.1007/s11004-017-9677-y}{doi:
  10.1007/s11004-017-9677-y}).

\bibitem[Kumar and Das, 1977]{GP.77.Kumar}
Kumar, R., and U.~C. Das,  1977, Transformation of dipole to {S}chlumberger
  sounding curves by means of digital linear filters: Geophysical Prospecting,
  {\bf 25}, 780--789.
\newblock (\href{https://doi.org/10.1111/j.1365-2478.1977.tb01204.x}{doi:
  10.1111/j.1365-2478.1977.tb01204.x}).

\bibitem[Kumar and Das, 1978]{GP.78.Kumar}
--------, 1978, Transformation of {S}chlumberger apparent resistivity to dipole
  apparent resistivity over layered earth by the application of digital linear
  filters: Geophysical Prospecting, {\bf 26}, 352--358.
\newblock (\href{https://doi.org/10.1111/j.1365-2478.1978.tb01598.x}{doi:
  10.1111/j.1365-2478.1978.tb01598.x}).

\bibitem[Mohsen and Hashish, 1994]{GP.94.Mohsen}
Mohsen, A.~A., and E.~A. Hashish,  1994, The fast {H}ankel transform:
  Geophysical Prospecting, {\bf 42}, 131--139.
\newblock (\href{https://doi.org/10.1111/j.1365-2478.1994.tb00202.x}{doi:
  10.1111/j.1365-2478.1994.tb00202.x}).

\bibitem[Murakami and Uchida, 1982]{GEO.1982.Murakami}
Murakami, Y., and T. Uchida,  1982, Accuracy of the linear filter coefficients
  determined by the iteration of the least-squares method: Geophysics, {\bf
  47}, 244--256.
\newblock (\href{https://doi.org/10.1190/1.1441331}{doi: 10.1190/1.1441331}).

\bibitem[Newman et~al., 1986]{GEO.86.Newman}
Newman, G.~A., G.~W. Hohmann, and W.~L. Anderson,  1986, Transient
  electromagnetic response of a three-dimensional body in a layered earth:
  Geophysics, {\bf 51}, 1608--1627.
\newblock (\href{https://doi.org/10.1190/1.1442212}{doi: 10.1190/1.1442212}).

\bibitem[Nissen and Enmark, 1986]{GP.86.Nissen}
Nissen, J., and T. Enmark,  1986, An optimized digital filter for the {F}ourier
  transform: Geophysical Prospecting, {\bf 34}, 897--903.
\newblock (\href{https://doi.org/10.1111/j.1365-2478.1986.tb00500.x}{doi:
  10.1111/j.1365-2478.1986.tb00500.x}).

\bibitem[Niwas, 1975]{GEO.75.Niwas}
Niwas, S.,  1975, Direct interpretation of geoelectric measurements by the use
  of linear filter theory: Geophysics, {\bf 40}, 121--122.
\newblock (\href{https://doi.org/10.1190/1.1440510}{doi: 10.1190/1.1440510}).

\bibitem[Nyman and Landisman, 1977]{GEO.77.Nyman}
Nyman, D.~C., and M. Landisman,  1977, {VES} dipole-dipole filter coefficients:
  Geophysics, {\bf 42}, 1037--1044.
\newblock (\href{https://doi.org/10.1190/1.1440763}{doi: 10.1190/1.1440763}).

\bibitem[O'Neill, 1975]{EXG.75.ONeill}
O'Neill, D.,  1975, Improved linear filter coefficients for application in
  apparent resistivity computations: Exploration Geophysics, {\bf 6}, 104--109.
\newblock (\href{https://doi.org/10.1071/EG975104}{doi: 10.1071/EG975104}).

\bibitem[O'Neill and Merrick, 1984]{GP.84.ONeill}
O'Neill, D.~J., and N.~P. Merrick,  1984, A digital linear filter for
  resistivity sounding with a generalized electrode array: Geophysical
  Prospecting, {\bf 32}, 105--123.
\newblock (\href{https://doi.org/10.1111/j.1365-2478.1984.tb00720.x}{doi:
  10.1111/j.1365-2478.1984.tb00720.x}).

\bibitem[Pekeris, 1940]{GEO.40.Pekeris}
Pekeris, C.~L.,  1940, Direct method of interpretation in resistivity
  prospecting: Geophysics, {\bf 5}, 31--42.
\newblock (\href{https://doi.org/10.1190/1.1441791}{doi: 10.1190/1.1441791}).

\bibitem[Raiche et~al., 2007]{ASEG.07.Raiche}
Raiche, A., F. Sugeng, and G. Wilson,  2007, Practical {3D} {EM} inversion -
  the {P223F} software suite: ASEG Technical Program Expanded Abstracts,  1--4.
\newblock (\href{https://doi.org/10.1071/ASEG2007ab114}{doi:
  10.1071/ASEG2007ab114}).

\bibitem[Slichter, 1933]{PHY.33.Slichter}
Slichter, L.~B.,  1933, The interpretation of the resistivity prospecting
  method for horizontal structures: Physics, {\bf 4}, 307--322.
\newblock (\href{https://doi.org/10.1063/1.1745198}{doi : 10.1063/1.1745198}).

\bibitem[Sørensen and Christensen, 1994]{GEO.94.Sorensen}
Sørensen, K.~I., and N.~B. Christensen,  1994, The fields from a finite
  electrical dipole—a new computational approach: Geophysics, {\bf 59},
  864--880.
\newblock (\href{https://doi.org/10.1190/1.1443646}{10.1190/1.1443646}).

\bibitem[Verma and Koefoed, 1973]{GP.73.Verma}
Verma, R.~K., and O. Koefoed,  1973, A note on the linear filter method of
  computing electromagnetic sounding curves: Geophysical Prospecting, {\bf 21},
  70--76.
\newblock (\href{https://doi.org/10.1111/j.1365-2478.1973.tb00015.x}{doi:
  10.1111/j.1365-2478.1973.tb00015.x}).

\bibitem[Ward and Hohmann, 1988]{B.SEG.88.Ward}
Ward, S.~H., and G.~W. Hohmann,  1988, Electromagnetic theory for geophysical
applications: SEG, Investigations in Geophysics, No.~3, 4, 130--131.
\newblock  (\href{https://doi.org/10.1190/1.9781560802631}{doi:
  10.1190/1.9781560802631}).

\bibitem[Werthmüller, 2017]{GEO.17.Werthmuller}
Werthmüller, D., 2017, An open-source full {3D} electromagnetic modeler for
  {1D} {VTI} media in {P}ython: empymod: Geophysics, {\bf 82}, no. 6,
  WB9--WB19.
\newblock (\href{https://doi.org/10.1190/geo2016-0626.1}{doi:
  10.1190/geo2016-0626.1}).

\bibitem[Werthmüller et~al., 2018]{GEO.18.Werthmuller} Werthmüller, D., K. Key,
  and E. Slob, 2018, A tool for designing digital filters for the Hankel and
  Fourier transforms in potential, diffusive, and wavefield modeling:
  Geophysics, {\bf 83}, no. ?, ????--????.
\newblock (\href{https://doi.org/10.1190/geo2018-0069.1}{doi:
  10.1190/geo2018-0069.1}).

\end{thebibliography}

\end{multicols}
\end{document}
